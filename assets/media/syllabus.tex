\documentclass[12pt]{article}
\usepackage{amsmath}
\usepackage{amssymb}
\usepackage{physics}
\usepackage{geometry}
\newgeometry{top = 2.5cm, bottom = 2cm, left=3cm, right=2.5cm}
\title{Advanced Computational Methods\\
Syllabus}
\begin{document}
\maketitle
This is a very hands-on course which will involve a lot of programming assignments. The
main aims of the course are two fold:\\
\textit{
1. Learning methods, tools and techniques for solving advanced scientific problems.\\
2. Developing practical computational problem solving skills.\\
}
\textbf{Textbooks:}\\

1. Mark Newman, \textit{Computational Physics,} CreateSpace Independent Publishing Platform
(2013).\\
2. Forman Acton, \textit{Real computing made real: Preventing Errors in Scientific and Engineering Calculations,} Dover Publications.\\
3. Lloyd N. Trefethen and David Bau, \textit{Numerical Linear Algebra,} SIAM.\\
4. William H. Press, Saul A. Teukolsky, William T. Vetterling and Brian P. Flannery,
\textit{Numerical Recipes 3rd Edition: The Art of Scientific Computing
}
\\
\textbf{1
Introduction to computational physics, computer architecture overview, tools
of computational physics (3 hours)}\\

What is computational physics? Why do we need it?; Computer hardware: basic computer
architecture, hierarchical memory, cache, latency and bandwidth; Moore’s law, power bottleneck; Software: compiled (Fortran, C) vs. interpreted languages (MATLAB, python);
software management.; Parallelization: MPI; OpenMP.\\

\textbf{2
Machine representation, precision and errors (1.5 hours)}\\

Representation on a computer: Integer representation; floating-point representation; Ma-
chine precision; Errors: round-off; approximation errors; random errors; errors of the third
kind; Quadratic equations; Power series; Delicate numerical expressions; Dangerous subtractions; Preserving small numbers; Partial Fractions; Cubic equations; Sketching functions;
\\
\textbf{3
Quadrature and Derivatives (6 hours)}\\

Direct fit polynomials; Quadrature methods on equal subintervals; Newton-Cotes formula;
Romberg Extrapolation; Gaussian quadrature; Adaptive step size; Special cases;
\\

\textbf{4
Solutions of linear and non-linear equations (9 hours)}

Simultaneous linear equations: Gauss elimination (pivoting, scaling); LU factorization; Cal-
culating inverse; Tri-diagonal systems; Eigenvalues and Eigenvectors: QR Factorization;
Gram-Schmidt Orthogonalization; Real roots of single variable function; Relaxation method;
qualitative behavior of the function; Closed domain methods (bracketing): Bisection; False
position method; Open domain methods: Newton-Raphson, Secant method; Complications;
Roots of polynomials; Roots of non-linear equations;
\\

\textbf{5
Fourier methods (3 hours)}

Fast Fourier transform; Convolution; Correlation; Power spectrum;
\\

\textbf{6
Random numbers and Monte-Carlo (6 hours)}\\
Random number generators; Monte-Carlo integration; Non-uniform distribution; Random
Walk; Metropolis algorithm;
\\

\textbf{7
Ordinary differential equations (9 hours)}

Initial value problems: First order Euler method; Second order single point methods; Runge-
Kutta methods; Multipoint methods; Boundary value problems: Shooting method; equilibrium boundary value method;
\\
\end{document}